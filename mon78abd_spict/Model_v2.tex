% Options for packages loaded elsewhere
\PassOptionsToPackage{unicode}{hyperref}
\PassOptionsToPackage{hyphens}{url}
%
\documentclass[
]{article}
\usepackage{lmodern}
\usepackage{amssymb,amsmath}
\usepackage{ifxetex,ifluatex}
\ifnum 0\ifxetex 1\fi\ifluatex 1\fi=0 % if pdftex
  \usepackage[T1]{fontenc}
  \usepackage[utf8]{inputenc}
  \usepackage{textcomp} % provide euro and other symbols
\else % if luatex or xetex
  \usepackage{unicode-math}
  \defaultfontfeatures{Scale=MatchLowercase}
  \defaultfontfeatures[\rmfamily]{Ligatures=TeX,Scale=1}
\fi
% Use upquote if available, for straight quotes in verbatim environments
\IfFileExists{upquote.sty}{\usepackage{upquote}}{}
\IfFileExists{microtype.sty}{% use microtype if available
  \usepackage[]{microtype}
  \UseMicrotypeSet[protrusion]{basicmath} % disable protrusion for tt fonts
}{}
\makeatletter
\@ifundefined{KOMAClassName}{% if non-KOMA class
  \IfFileExists{parskip.sty}{%
    \usepackage{parskip}
  }{% else
    \setlength{\parindent}{0pt}
    \setlength{\parskip}{6pt plus 2pt minus 1pt}}
}{% if KOMA class
  \KOMAoptions{parskip=half}}
\makeatother
\usepackage{xcolor}
\IfFileExists{xurl.sty}{\usepackage{xurl}}{} % add URL line breaks if available
\IfFileExists{bookmark.sty}{\usepackage{bookmark}}{\usepackage{hyperref}}
\hypersetup{
  hidelinks,
  pdfcreator={LaTeX via pandoc}}
\urlstyle{same} % disable monospaced font for URLs
\usepackage[margin=1in]{geometry}
\usepackage{color}
\usepackage{fancyvrb}
\newcommand{\VerbBar}{|}
\newcommand{\VERB}{\Verb[commandchars=\\\{\}]}
\DefineVerbatimEnvironment{Highlighting}{Verbatim}{commandchars=\\\{\}}
% Add ',fontsize=\small' for more characters per line
\usepackage{framed}
\definecolor{shadecolor}{RGB}{248,248,248}
\newenvironment{Shaded}{\begin{snugshade}}{\end{snugshade}}
\newcommand{\AlertTok}[1]{\textcolor[rgb]{0.94,0.16,0.16}{#1}}
\newcommand{\AnnotationTok}[1]{\textcolor[rgb]{0.56,0.35,0.01}{\textbf{\textit{#1}}}}
\newcommand{\AttributeTok}[1]{\textcolor[rgb]{0.77,0.63,0.00}{#1}}
\newcommand{\BaseNTok}[1]{\textcolor[rgb]{0.00,0.00,0.81}{#1}}
\newcommand{\BuiltInTok}[1]{#1}
\newcommand{\CharTok}[1]{\textcolor[rgb]{0.31,0.60,0.02}{#1}}
\newcommand{\CommentTok}[1]{\textcolor[rgb]{0.56,0.35,0.01}{\textit{#1}}}
\newcommand{\CommentVarTok}[1]{\textcolor[rgb]{0.56,0.35,0.01}{\textbf{\textit{#1}}}}
\newcommand{\ConstantTok}[1]{\textcolor[rgb]{0.00,0.00,0.00}{#1}}
\newcommand{\ControlFlowTok}[1]{\textcolor[rgb]{0.13,0.29,0.53}{\textbf{#1}}}
\newcommand{\DataTypeTok}[1]{\textcolor[rgb]{0.13,0.29,0.53}{#1}}
\newcommand{\DecValTok}[1]{\textcolor[rgb]{0.00,0.00,0.81}{#1}}
\newcommand{\DocumentationTok}[1]{\textcolor[rgb]{0.56,0.35,0.01}{\textbf{\textit{#1}}}}
\newcommand{\ErrorTok}[1]{\textcolor[rgb]{0.64,0.00,0.00}{\textbf{#1}}}
\newcommand{\ExtensionTok}[1]{#1}
\newcommand{\FloatTok}[1]{\textcolor[rgb]{0.00,0.00,0.81}{#1}}
\newcommand{\FunctionTok}[1]{\textcolor[rgb]{0.00,0.00,0.00}{#1}}
\newcommand{\ImportTok}[1]{#1}
\newcommand{\InformationTok}[1]{\textcolor[rgb]{0.56,0.35,0.01}{\textbf{\textit{#1}}}}
\newcommand{\KeywordTok}[1]{\textcolor[rgb]{0.13,0.29,0.53}{\textbf{#1}}}
\newcommand{\NormalTok}[1]{#1}
\newcommand{\OperatorTok}[1]{\textcolor[rgb]{0.81,0.36,0.00}{\textbf{#1}}}
\newcommand{\OtherTok}[1]{\textcolor[rgb]{0.56,0.35,0.01}{#1}}
\newcommand{\PreprocessorTok}[1]{\textcolor[rgb]{0.56,0.35,0.01}{\textit{#1}}}
\newcommand{\RegionMarkerTok}[1]{#1}
\newcommand{\SpecialCharTok}[1]{\textcolor[rgb]{0.00,0.00,0.00}{#1}}
\newcommand{\SpecialStringTok}[1]{\textcolor[rgb]{0.31,0.60,0.02}{#1}}
\newcommand{\StringTok}[1]{\textcolor[rgb]{0.31,0.60,0.02}{#1}}
\newcommand{\VariableTok}[1]{\textcolor[rgb]{0.00,0.00,0.00}{#1}}
\newcommand{\VerbatimStringTok}[1]{\textcolor[rgb]{0.31,0.60,0.02}{#1}}
\newcommand{\WarningTok}[1]{\textcolor[rgb]{0.56,0.35,0.01}{\textbf{\textit{#1}}}}
\usepackage{graphicx,grffile}
\makeatletter
\def\maxwidth{\ifdim\Gin@nat@width>\linewidth\linewidth\else\Gin@nat@width\fi}
\def\maxheight{\ifdim\Gin@nat@height>\textheight\textheight\else\Gin@nat@height\fi}
\makeatother
% Scale images if necessary, so that they will not overflow the page
% margins by default, and it is still possible to overwrite the defaults
% using explicit options in \includegraphics[width, height, ...]{}
\setkeys{Gin}{width=\maxwidth,height=\maxheight,keepaspectratio}
% Set default figure placement to htbp
\makeatletter
\def\fps@figure{htbp}
\makeatother
\setlength{\emergencystretch}{3em} % prevent overfull lines
\providecommand{\tightlist}{%
  \setlength{\itemsep}{0pt}\setlength{\parskip}{0pt}}
\setcounter{secnumdepth}{-\maxdimen} % remove section numbering

\author{}
\date{\vspace{-2.5em}}

\begin{document}

\addtolength{\headheight}{1.0cm} 
\pagestyle{fancyplain}

\renewcommand{\headrulewidth}{0pt}

\hypertarget{aim}{%
\section{AIM}\label{aim}}

The aim is to try spict as assessment model for Northern White
Anglerfish in Division 7a-k and compare the results with the outputs of
the actual assessment model a4a. The a4a model get information from
three surveys:

FR\_IE\_IBTS survey: the joint index of the French EVHOE survey and the
Irish IBTS survey. Both are conducted between the third and forth
quarter.The French EVHOE survey is conducted in the Bay of Biscay and
the Irish IBTS in the Celtic Sea.

The Spanish survey in the Porcupine Bank began in 2001 and covers ICES
Divisions VIIb-k corresponding to the Porcupine Bank and adjacent area
in western Irish waters. The survey takes place in the third quarter
(September) and covers depths between 170 and 800 m.

Monkfish survey is in the beginning of the year, but is considered the
end of the previous year and it's conducted in the Celtic Sea.

In addition of the three surveys,here we also include the standarized
LPUE from the trawlers of Vigo.

\begin{Shaded}
\begin{Highlighting}[]
\KeywordTok{library}\NormalTok{(devtools)}
\KeywordTok{install_github}\NormalTok{(}\StringTok{"DTUAqua/spict/spict"}\NormalTok{) }
\end{Highlighting}
\end{Shaded}

Load all necessary packages.

\begin{Shaded}
\begin{Highlighting}[]
\KeywordTok{library}\NormalTok{(spict)}
\end{Highlighting}
\end{Shaded}

\hypertarget{input-data}{%
\section{Input data}\label{input-data}}

\begin{Shaded}
\begin{Highlighting}[]
\NormalTok{data.ang <-}\StringTok{ }\KeywordTok{read.csv}\NormalTok{(}\StringTok{"data_Spict.csv"}\NormalTok{)}
\KeywordTok{summary}\NormalTok{(data.ang)}

\NormalTok{data.ang}\OperatorTok{$}\NormalTok{Year <-}\StringTok{ }\KeywordTok{as.numeric}\NormalTok{(data.ang}\OperatorTok{$}\NormalTok{Year)}
\NormalTok{data.ang}\OperatorTok{$}\NormalTok{timeEv <-}\StringTok{ }\KeywordTok{ifelse}\NormalTok{(data.ang}\OperatorTok{$}\NormalTok{Year}\OperatorTok{>=}\DecValTok{2003} \OperatorTok{&}\StringTok{ }\NormalTok{data.ang}\OperatorTok{$}\NormalTok{Year}\OperatorTok{!=}\StringTok{ }\DecValTok{2017}\NormalTok{,data.ang}\OperatorTok{$}\NormalTok{Year}\OperatorTok{+}\NormalTok{(}\FloatTok{0.75}\OperatorTok{+}\DecValTok{1}\NormalTok{)}\OperatorTok{/}\DecValTok{2}\NormalTok{,}\OtherTok{NA}\NormalTok{)}
\NormalTok{data.ang}\OperatorTok{$}\NormalTok{timePP <-}\StringTok{ }\KeywordTok{ifelse}\NormalTok{(data.ang}\OperatorTok{$}\NormalTok{Year}\OperatorTok{>=}\DecValTok{2001}\NormalTok{,data.ang}\OperatorTok{$}\NormalTok{Year}\OperatorTok{+}\NormalTok{(}\FloatTok{0.75}\OperatorTok{+}\DecValTok{1}\NormalTok{)}\OperatorTok{/}\DecValTok{2}\NormalTok{,}\OtherTok{NA}\NormalTok{)}
\NormalTok{data.ang}\OperatorTok{$}\NormalTok{timeMon <-}\StringTok{ }\KeywordTok{ifelse}\NormalTok{(data.ang}\OperatorTok{$}\NormalTok{Year }\OperatorTok\StringTok{ }\KeywordTok{c}\NormalTok{(}\DecValTok{2005}\NormalTok{,}\DecValTok{2006}\NormalTok{,}\DecValTok{2015}\OperatorTok{:}\DecValTok{2019}\NormalTok{),data.ang}\OperatorTok{$}\NormalTok{Year}\OperatorTok{+}\DecValTok{1}\OperatorTok{/}\DecValTok{12}\OperatorTok{*}\FloatTok{11.5}\NormalTok{,}\OtherTok{NA}\NormalTok{)}
\NormalTok{data.ang}\OperatorTok{$}\NormalTok{timeMon2 <-}\StringTok{ }\KeywordTok{ifelse}\NormalTok{(data.ang}\OperatorTok{$}\NormalTok{Year }\OperatorTok{>=}\StringTok{ }\DecValTok{2015}\NormalTok{, data.ang}\OperatorTok{$}\NormalTok{Year}\OperatorTok{+}\DecValTok{1}\OperatorTok{/}\DecValTok{12}\OperatorTok{*}\FloatTok{11.5}\NormalTok{,}\OtherTok{NA}\NormalTok{)}

\KeywordTok{attach}\NormalTok{(data.ang)}
\end{Highlighting}
\end{Shaded}

\begin{Shaded}
\begin{Highlighting}[]
\NormalTok{dtc=}\DecValTok{1}
\NormalTok{dteuler=}\DecValTok{1}\OperatorTok{/}\DecValTok{16}
\NormalTok{inp.ang <-}\StringTok{ }\KeywordTok{list}\NormalTok{(}\DataTypeTok{obsC=}\NormalTok{Total_Catch,}\DataTypeTok{timeC=}\NormalTok{Year,}
                \DataTypeTok{obsI=}\KeywordTok{list}\NormalTok{(LPUE_Vigo, SPPGFS,MON2),}
                \DataTypeTok{timeI=}\KeywordTok{list}\NormalTok{(Year,timePP,timeMon2),}
                \DataTypeTok{dtc=}\NormalTok{dtc,}\DataTypeTok{dteuler=}\NormalTok{dteuler)}
\NormalTok{inp.ang}\OperatorTok{$}\NormalTok{phases}\OperatorTok{$}\NormalTok{logbeta <-}\StringTok{ }\DecValTok{1}
\NormalTok{inp.ang}\OperatorTok{$}\NormalTok{phases}\OperatorTok{$}\NormalTok{logalpha <-}\StringTok{ }\DecValTok{1}
\NormalTok{inp <-}\StringTok{ }\KeywordTok{check.inp}\NormalTok{(inp.ang)}
\CommentTok{#' Plot the input data}
\KeywordTok{plotspict.data}\NormalTok{(inp, }\DataTypeTok{qlegend =} \OtherTok{TRUE}\NormalTok{)}
\end{Highlighting}
\end{Shaded}

\includegraphics{Model_v2_files/figure-latex/unnamed-chunk-3-1.pdf}

\begin{Shaded}
\begin{Highlighting}[]
\NormalTok{res <-}\StringTok{ }\KeywordTok{fit.spict}\NormalTok{(inp)}
\KeywordTok{summary}\NormalTok{(res)}
\NormalTok{res<-}\StringTok{ }\KeywordTok{calc.osa.resid}\NormalTok{(res)}

\KeywordTok{plot}\NormalTok{(res)}
\end{Highlighting}
\end{Shaded}

\includegraphics{Model_v2_files/figure-latex/unnamed-chunk-3-2.pdf}

\begin{Shaded}
\begin{Highlighting}[]
\CommentTok{#plotspict.osar(rep)}


\CommentTok{# Plot the residual diagnostics}
\KeywordTok{plotspict.diagnostic}\NormalTok{(res)}
\end{Highlighting}
\end{Shaded}

\includegraphics{Model_v2_files/figure-latex/unnamed-chunk-3-3.pdf}

\begin{Shaded}
\begin{Highlighting}[]
\CommentTok{# Check the sensitivity to initial parameter values}
\NormalTok{sens.ini <-}\StringTok{ }\KeywordTok{check.ini}\NormalTok{(res, }\DataTypeTok{ntrials =} \DecValTok{5}\NormalTok{)}

\CommentTok{# Make a retrospective analysis and plot the results}
\NormalTok{res}\OperatorTok{$}\NormalTok{inp}\OperatorTok{$}\NormalTok{getReportCovariance =}\StringTok{ }\OtherTok{FALSE}
\NormalTok{res <-}\StringTok{ }\KeywordTok{retro}\NormalTok{(res, }\DataTypeTok{nretroyear =} \DecValTok{4}\NormalTok{)}
\KeywordTok{plotspict.retro}\NormalTok{(res)}
\end{Highlighting}
\end{Shaded}

\includegraphics{Model_v2_files/figure-latex/unnamed-chunk-3-4.pdf}

\begin{Shaded}
\begin{Highlighting}[]
\CommentTok{# Plot the relative biomass and fishing mortality}
\KeywordTok{par}\NormalTok{(}\DataTypeTok{mfrow =} \KeywordTok{c}\NormalTok{(}\DecValTok{2}\NormalTok{, }\DecValTok{1}\NormalTok{), }\DataTypeTok{mar =} \KeywordTok{c}\NormalTok{(}\DecValTok{4}\NormalTok{, }\DecValTok{4}\NormalTok{, }\DecValTok{1}\NormalTok{, }\DecValTok{1}\NormalTok{))}
\KeywordTok{plotspict.bbmsy}\NormalTok{(res, }\DataTypeTok{qlegend =} \OtherTok{FALSE}\NormalTok{, }\DataTypeTok{stamp =} \StringTok{""}\NormalTok{)}
\KeywordTok{plotspict.ffmsy}\NormalTok{(res, }\DataTypeTok{qlegend =} \OtherTok{FALSE}\NormalTok{, }\DataTypeTok{stamp =} \StringTok{""}\NormalTok{)}
\end{Highlighting}
\end{Shaded}

\includegraphics{Model_v2_files/figure-latex/unnamed-chunk-3-5.pdf}

\begin{Shaded}
\begin{Highlighting}[]
\CommentTok{# Check if there are changes in the results if the grid is finer, i.e.}
\CommentTok{# dteuler is smaller than 1/16 year.}
\NormalTok{inp.dteuler1_}\DecValTok{32}\NormalTok{ <-}\StringTok{ }\NormalTok{inp.ang}
\NormalTok{inp.dteuler1_}\DecValTok{32}\OperatorTok{$}\NormalTok{dteuler <-}\StringTok{ }\DecValTok{1}\OperatorTok{/}\DecValTok{32}
\NormalTok{inp.dteuler1_}\DecValTok{32}\NormalTok{ <-}\StringTok{ }\KeywordTok{check.inp}\NormalTok{(inp.dteuler1_}\DecValTok{32}\NormalTok{)}

\NormalTok{fit.dteuler1_}\DecValTok{32}\NormalTok{ <-}\StringTok{ }\KeywordTok{fit.spict}\NormalTok{(inp.dteuler1_}\DecValTok{32}\NormalTok{)}

\CommentTok{## Fit the discrete time model}
\NormalTok{inp.dteuler1 <-}\StringTok{ }\NormalTok{inp.ang}
\NormalTok{inp.dteuler1}\OperatorTok{$}\NormalTok{dteuler <-}\StringTok{ }\DecValTok{1}
\NormalTok{inp.dteuler1 <-}\StringTok{ }\KeywordTok{check.inp}\NormalTok{(inp.dteuler1)}

\NormalTok{fit.dteuler1 <-}\StringTok{ }\KeywordTok{fit.spict}\NormalTok{(inp.dteuler1)}

\CommentTok{# Checked the fixed effects estimates}
\NormalTok{(par <-}\StringTok{ }\KeywordTok{exp}\NormalTok{(res}\OperatorTok{$}\NormalTok{par.fixed))}
\NormalTok{(par1 <-}\StringTok{ }\KeywordTok{exp}\NormalTok{(fit.dteuler1}\OperatorTok{$}\NormalTok{par.fixed))}
\NormalTok{(par1_}\DecValTok{32}\NormalTok{ <-}\StringTok{ }\KeywordTok{exp}\NormalTok{(fit.dteuler1_}\DecValTok{32}\OperatorTok{$}\NormalTok{par.fixed))}

\CommentTok{# Plot the percent change compared to the default run}
\NormalTok{dif1_}\DecValTok{32}\NormalTok{ <-}\StringTok{ }\NormalTok{(par }\OperatorTok{-}\StringTok{ }\NormalTok{par1_}\DecValTok{32}\NormalTok{) }\OperatorTok{/}\StringTok{ }\NormalTok{par }\OperatorTok{*}\StringTok{ }\DecValTok{100}
\NormalTok{dif1 <-}\StringTok{ }\NormalTok{(par }\OperatorTok{-}\StringTok{ }\NormalTok{par1) }\OperatorTok{/}\StringTok{ }\NormalTok{par }\OperatorTok{*}\StringTok{ }\DecValTok{100}
\KeywordTok{plot}\NormalTok{(dif1, }\DataTypeTok{ylab =} \StringTok{"Percent difference"}\NormalTok{, }\DataTypeTok{xlab =} \StringTok{""}\NormalTok{, }\DataTypeTok{axes =} \OtherTok{FALSE}\NormalTok{, }\DataTypeTok{ylim =} \KeywordTok{c}\NormalTok{(}\OperatorTok{-}\DecValTok{40}\NormalTok{,}\DecValTok{40}\NormalTok{))}
\KeywordTok{points}\NormalTok{(dif1_}\DecValTok{32}\NormalTok{, }\DataTypeTok{pch =} \DecValTok{3}\NormalTok{)}
\KeywordTok{abline}\NormalTok{(}\DataTypeTok{h=}\DecValTok{0}\NormalTok{, }\DataTypeTok{col =} \StringTok{"#00000044"}\NormalTok{)}
\KeywordTok{axis}\NormalTok{(}\DecValTok{1}\NormalTok{, }\DataTypeTok{at=}\KeywordTok{seq}\NormalTok{(par), }\DataTypeTok{labels =} \KeywordTok{gsub}\NormalTok{(}\StringTok{"log"}\NormalTok{, }\StringTok{""}\NormalTok{, }\KeywordTok{names}\NormalTok{(par)))}
\KeywordTok{axis}\NormalTok{(}\DecValTok{2}\NormalTok{, }\DataTypeTok{at =} \KeywordTok{seq}\NormalTok{(}\OperatorTok{-}\DecValTok{30}\NormalTok{, }\DecValTok{30}\NormalTok{, }\DecValTok{20}\NormalTok{))}
\KeywordTok{legend}\NormalTok{(}\StringTok{"topleft"}\NormalTok{, }\DataTypeTok{legend =} \KeywordTok{c}\NormalTok{(}\StringTok{"Discrete"}\NormalTok{, }\StringTok{"Finer grid"}\NormalTok{), }\DataTypeTok{pch =} \KeywordTok{c}\NormalTok{(}\DecValTok{1}\NormalTok{,}\DecValTok{3}\NormalTok{))}

\CommentTok{# Plot the results using the two dteuler together}
\KeywordTok{par}\NormalTok{(}\DataTypeTok{mfrow =} \KeywordTok{c}\NormalTok{(}\DecValTok{2}\NormalTok{,}\DecValTok{3}\NormalTok{))}
\end{Highlighting}
\end{Shaded}

\includegraphics{Model_v2_files/figure-latex/unnamed-chunk-3-6.pdf}

\begin{Shaded}
\begin{Highlighting}[]
\KeywordTok{plotspict.bbmsy}\NormalTok{(res, }\DataTypeTok{qlegend =} \OtherTok{FALSE}\NormalTok{, }\DataTypeTok{main =} \StringTok{""}\NormalTok{)}
\KeywordTok{title}\NormalTok{(}\StringTok{"Default: dteuler = 1/16 year"}\NormalTok{)}
\KeywordTok{plotspict.bbmsy}\NormalTok{(fit.dteuler1_}\DecValTok{32}\NormalTok{, }\DataTypeTok{qlegend =} \OtherTok{FALSE}\NormalTok{, }\DataTypeTok{main =} \StringTok{""}\NormalTok{)}
\KeywordTok{title}\NormalTok{(}\StringTok{"Finer grid: dteuler = 1/32 year"}\NormalTok{)}
\KeywordTok{plotspict.bbmsy}\NormalTok{(fit.dteuler1, }\DataTypeTok{qlegend =} \OtherTok{FALSE}\NormalTok{, }\DataTypeTok{main =} \StringTok{""}\NormalTok{)}
\KeywordTok{title}\NormalTok{(}\StringTok{"Discrete time: dteuler = 1 year"}\NormalTok{)}
\KeywordTok{plotspict.ffmsy}\NormalTok{(res)}
\KeywordTok{plotspict.ffmsy}\NormalTok{(fit.dteuler1_}\DecValTok{32}\NormalTok{)}
\KeywordTok{plotspict.ffmsy}\NormalTok{(fit.dteuler1)}
\end{Highlighting}
\end{Shaded}

\includegraphics{Model_v2_files/figure-latex/unnamed-chunk-3-7.pdf}

\end{document}
